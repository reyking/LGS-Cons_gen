\documentclass{article}
\usepackage[margin=2cm]{geometry}
\usepackage{hhline} % For highlighting table borders
\usepackage{hyperref}
\usepackage{array}
\usepackage{xcolor}
\usepackage{longtable,makecell,multirow}
\begin{document}
\section*{Resumen de actividades realizadas agosto/2023}

En el siguiente reporte se presentan las actividades realizadas por Hans Chritopher Raddatz Garcia durante el mes de agosto del 2023, donde se numeran las actividades principales dentro de la tabla de "Historias" con su total de horas, según las tarifas previamente acordadas, estas tareas principales se subdividen en cada trabajo realizado y estos son descritos en la tabla de "Tareas y detalles" indicando una descripción breve de lo realizado, las personas que estuvieron presente si es que las hubo y el tiempo en horas según cada tarifa.
Finalmente, se presenta las simbologías de tarifas y personas, además cada texto en azul implica que es un vínculo a su respectiva definición dentro de otra tabla, el cual además es clickeable. 

\section{Historias desarrolladas}
En la siguiente tabla se presentan las historias o desarrollos principales con su nombre y descripción y su tiempo total por cada tipo de tarifa.

\begin{table}[htbp]
    \centering
    \begin{tabular}{|c|p{3cm}|p{5.8cm}|m{4.5cm}|}
        \hline
        \textbf{N°} & \centering{\textbf{Nombre Historia}} & \centering{\textbf{Detalle}} & \textbf{Tiempo Total [hrs]}
        \\ \hline
  \label{1}1  & Codigos de error API-PAY & Restructurar los codigos de error que envia api-pay en cada endpoint tal que coincida con el mensaje y razon del error &
            \begin{tabular}{m{4cm}}
Desarrollo = 1 \\ 

            \end{tabular} 
            \\ \hline
  \label{2}2  & Review-retro planing & reunion bimensual para oranizar tareas del sprint &
            \begin{tabular}{m{4cm}}
Desarrollo = 2 \\ 
\hline \hline
Soporte y reuniones = 1.5 \\ 

            \end{tabular} 
            \\ \hline
  \label{4}4  & onAir para api-ott/admin & generar atributo y control en admin para estado de "al aire" en las series &
            \begin{tabular}{m{4cm}}
Soporte y reuniones horario inhábil = 3 \\ 

            \end{tabular} 
            \\ \hline
  \label{6}6  & Hotfix, revision y solucion de errores & Solucion de problemas no programados en historias &
            \begin{tabular}{m{4cm}}
Soporte y reuniones = 3.5 \\ 
\hline \hline
Desarrollo = 3 \\ 
\hline \hline
Soporte y reuniones horario inhábil = 1.5 \\ 

            \end{tabular} 
            \\ \hline
  \label{7}7  & Revisión de ventas sin registrar en Odoo & Revisión de ventas sin registrar en Odoo para despubrir proiblemas de implementacion act ual en api pay &
            \begin{tabular}{m{4cm}}
Soporte y reuniones = 4 \\ 

            \end{tabular} 
            \\ \hline
  \label{8}8  & Reuniones de sincronizacion & Reuniones de sincronizacion no asocidas a alguna historia en particular &
            \begin{tabular}{m{4cm}}
Soporte y reuniones = 1 \\ 

            \end{tabular} 
            \\ \hline
  \label{9}9  & Problemas de duplicidad de subscripciones & Corregir los problemas encontrados que pueden generar multiplicidad de subscripciones de pago para los clientes &
            \begin{tabular}{m{4cm}}
Desarrollo = 7 \\ 
\hline \hline
Soporte y reuniones = 2.5 \\ 

            \end{tabular} 
            \\ \hline
  \label{10}10  & Dias trial mas 1 & Agregar 1 dia a trial para cumplir con oferta &
            \begin{tabular}{m{4cm}}
Soporte y reuniones = 3.5 \\ 

            \end{tabular} 
            \\ \hline
    \end{tabular}
\end{table} 
 
\section{Tareas y detalles}
A continuación se presentan las tareas realizadas con su respectiva explicación, número de historia, un acrónimo de o las personas involucradas en la realización de la tarea si aplica y finalmente un detalle de horas totales trabajadas separado por cada tipo de tarifa.

\begin{longtable}{|m{0.5cm}|m{1.2cm}|p{5cm}|m{1.5cm}|m{1.5cm}||c|c|c|c||}
        \hline
        \multirow{2}{=}{\centering{\textbf{N°}}} & \multirow{2}{=}{\centering{\textbf{N°Hist}}} & \multirow{2}{=}{\centering{\textbf{Detalle Tarea}}}  & \multirow{2}{=}{\textbf{Personas}} & \multirow{2}{=}{\textbf{Fecha}} &   
        \multicolumn{4}{c|}{
            \textbf{Horas trabajadas [hrs]}
        } \\ 
        \hhline{~~~~~----}
        &&&&&  \hyperref[D]{\color{blue}D}  
&  \hyperref[SYR]{\color{blue}SYR}  
&  \hyperref[SYRHI]{\color{blue}SYRHI}  
&  \hyperref[P]{\color{blue}P}  
\\ \hline \hline
                 \label{0}0  &  \hyperref[1]{\color{blue}1}  & Incluir cambios realizados para merge y probar &  
                 & 13/09/23 08:00  & 1 &  &  & \\ \hline 

                 \label{1}1  &  \hyperref[2]{\color{blue}2}  & 	
review retro planing sprint 61 &  
                 & 01/09/23 12:00  & 2 &  &  & \\ \hline 

                 \label{2}2  &  \hyperref[2]{\color{blue}2}  & Review/retro planing sprint 62 &  
                 & 15/09/23 09:00  &  & 1.5 &  & \\ \hline 

                 \label{3}3  &  \hyperref[4]{\color{blue}4}  & Generar release para produccion de api-ott y admin &  
                 & 13/09/23 02:00  &  &  & 3 & \\ \hline 

                 \label{4}4  &  \hyperref[6]{\color{blue}6}  & Revisar cambios de paquetes para despliegue &  
                  \hyperref[L.Q.]{\color{blue}L.Q.} \newline  \hyperref[M.F.]{\color{blue}M.F.}  & 04/09/23 09:00  &  & 2.5 &  & \\ \hline 

                 \label{5}5  &  \hyperref[6]{\color{blue}6}  & Solucionar error asincrono de conexion a base de datos en historial de modificaciones &  
                 & 04/09/23 12:00  & 3 &  &  & \\ \hline 

                 \label{6}6  &  \hyperref[6]{\color{blue}6}  & Deploy de cambios en paquetes api-pay &  
                  \hyperref[L.Q.]{\color{blue}L.Q.} \newline  \hyperref[M.F.]{\color{blue}M.F.}  & 05/09/23 04:00  &  &  & 1.5 & \\ \hline 

                 \label{7}7  &  \hyperref[6]{\color{blue}6}  & Creacion de indices en produccion para postgres de api-pay &  
                 & 15/09/23 01:00  &  & 1 &  & \\ \hline 

                 \label{8}8  &  \hyperref[7]{\color{blue}7}  & Busqueda de problema de pago en clientes con quejas &  
                 & 05/09/23 07:00  &  & 4 &  & \\ \hline 

                 \label{9}9  &  \hyperref[8]{\color{blue}8}  & Reuniones de sinc &  
                 & 07/09/23 01:00  &  & 1 &  & \\ \hline 

                 \label{10}10  &  \hyperref[9]{\color{blue}9}  & validacion de estados y anulacion de trancciones cuando exista duplicidad previa a enrolamiento &  
                 & 11/09/23 12:00  & 3 &  &  & \\ \hline 

                 \label{11}11  &  \hyperref[9]{\color{blue}9}  & anular transacciones y susbcripciones duplicadas que llegen a enrolarse &  
                  \hyperref[M.F.]{\color{blue}M.F.}  & 13/09/23 09:00  &  & 1.5 &  & \\ \hline 

                 \label{12}12  &  \hyperref[9]{\color{blue}9}  & Generar release para produccion de api-pay &  
                 & 13/09/23 01:00  &  & 1 &  & \\ \hline 

                 \label{13}13  &  \hyperref[9]{\color{blue}9}  & Anulacion de multiplicidad de transacciones al confirmar tarjeta &  
                 & 20/09/23 11:00  & 4 &  &  & \\ \hline 

                 \label{14}14  &  \hyperref[10]{\color{blue}10}  & Revision +1 dia al trial y sus correciones &  
                 & 12/09/23 08:00  &  & 3.5 &  & \\ \hline 

    \end{longtable} 
 \newpage
\section{Personas}
Listado de personas y su abreviación para tareas en las que estuvieron involucrados.
\begin{table}[htbp]
    \centering
    \begin{tabular}{|p{6cm}|c|}
        \hline
        \centering{\textbf{Nombre}} & \textbf{abreviacion} \\ \hline
        Luiz Quelvez &  \label{L.Q.}L.Q.  \\ \hline
        Marco Farias &  \label{M.F.}M.F.  \\ \hline
    \end{tabular}
\end{table} 
 
\section{Tarifas}
    En la siguiente tabla se presentan las abreviaciones de cada tarifa.
\begin{table}[htbp]
    \centering
    \begin{tabular}{|p{6cm}|c|}
        \hline
        \centering{\textbf{Nombre}} & \textbf{abreviacion} \\ \hline  Desarrollo &  \label{D}D  \\ \hline 
 Soporte y reuniones &  \label{SYR}SYR  \\ \hline 
 Soporte y reuniones horario inhábil &  \label{SYRHI}SYRHI  \\ \hline 
 Presencial &  \label{P}P  \\ \hline 

    \end{tabular}
\end{table} 
 \end{document}