\documentclass{article}
\usepackage[margin=2cm]{geometry}
\usepackage{hhline} % For highlighting table borders
\usepackage{hyperref}
\usepackage{array}
\usepackage{xcolor}
\usepackage{longtable,makecell,multirow}
\begin{document}
\section*{Resumen de actividades realizadas diciembre/2023}

En el siguiente reporte se presentan las actividades realizadas por Hans Chritopher Raddatz Garcia durante el mes de diciembre del 2023, donde se numeran las actividades principales dentro de la tabla de "Historias" con su total de horas, según las tarifas previamente acordadas, estas tareas principales se subdividen en cada trabajo realizado y estos son descritos en la tabla de "Tareas y detalles" indicando una descripción breve de lo realizado, las personas que estuvieron presente si es que las hubo y el tiempo en horas según cada tarifa.
Finalmente, se presenta las simbologías de tarifas y personas, además cada texto en azul implica que es un vínculo a su respectiva definición dentro de otra tabla, el cual además es clickeable. 

\section{Historias desarrolladas}
En la siguiente tabla se presentan las historias o desarrollos principales con su nombre y descripción y su tiempo total por cada tipo de tarifa.

\begin{table}[htbp]
    \centering
    \begin{tabular}{|c|p{3cm}|p{5.8cm}|m{4.5cm}|}
        \hline
        \textbf{N°} & \centering{\textbf{Nombre Historia}} & \centering{\textbf{Detalle}} & \textbf{Tiempo Total [hrs]}
        \\ \hline
  \label{2}2  & Review-retro planing & reunion bimensual para oranizar tareas del sprint &
            \begin{tabular}{m{4cm}}
Desarrollos en conjunto = -2 \\ 
\hline \hline
Soporte y reuniones en conjunto = 2 \\ 

            \end{tabular} 
            \\ \hline
  \label{6}6  & Hotfix, revision y solucion de errores & Solucion de problemas no programados en historias &
            \begin{tabular}{m{4cm}}
Desarrollo = 3 \\ 

            \end{tabular} 
            \\ \hline
  \label{14}14  & Otras reuniones & cualquier otra reunion que no este directamente asociada a una historia, ejemplo incidencias y organizacion trabajos entre multiples areas &
            \begin{tabular}{m{4cm}}
Desarrollos en conjunto = 1.5 \\ 
\hline \hline
Desarrollos en conjunto en horario inhábil = 1 \\ 

            \end{tabular} 
            \\ \hline
  \label{18}18  & Estandarizar funcionalidad de cron & reordenar y estandarizar funciones para obteren los receptores de cambios y ejecucion sobre dichos receptores. Almacenar esta informacion en estructura de jobsEjecutions &
            \begin{tabular}{m{4cm}}
Desarrollo = 5 \\ 

            \end{tabular} 
            \\ \hline
  \label{19}19  & relacion cupon-plan en api-pay & Crear y asplicar realicon/restriccion entre cupones/planes &
            \begin{tabular}{m{4cm}}
Desarrollo = 3 \\ 

            \end{tabular} 
            \\ \hline
  \label{20}20  & Cambio de plan & flujo completo para realizar el cambio de un plan a otro por su periodicidad &
            \begin{tabular}{m{4cm}}
Desarrollo = 20 \\ 
\hline \hline
Desarrollos en conjunto = 4 \\ 

            \end{tabular} 
            \\ \hline
  \label{21}21  & Idp claro & endpints para obtener informacion del idp e integrar idp claro &
            \begin{tabular}{m{4cm}}
Desarrollo = 7 \\ 
\hline \hline
Desarrollos en conjunto en horario inhábil = 2.5 \\ 

            \end{tabular} 
            \\ \hline
  \label{23}23  & Dominios & Sistema que permita la creacion de susbcripciones por defecto segun el dominio de origen del correo del usuario y cron para mantener sincronizar diferencias &
            \begin{tabular}{m{4cm}}
Desarrollo = 23 \\ 

            \end{tabular} 
            \\ \hline
    \end{tabular}
\end{table} 
 
\section{Tareas y detalles}
A continuación se presentan las tareas realizadas con su respectiva explicación, número de historia, un acrónimo de o las personas involucradas en la realización de la tarea si aplica y finalmente un detalle de horas totales trabajadas separado por cada tipo de tarifa.

\begin{longtable}{|m{0.5cm}|m{1.2cm}|p{5cm}|m{1.5cm}|m{1.5cm}||c|c|c|c|c|c|    |}
        \hline
        \multirow{2}{=}{\centering{\textbf{N°}}} & \multirow{2}{=}{\centering{\textbf{N°Hist}}} & \multirow{2}{=}{\centering{\textbf{Detalle Tarea}}}  & \multirow{2}{=}{\textbf{Personas}} & \multirow{2}{=}{\textbf{Fecha}} &   
        \multicolumn{6}{c|}{
            \textbf{Horas trabajadas [hrs]}
        } \\ 
        \hhline{~~~~~----}
        &&&&&  \hyperref[D]{\color{blue}D}  
&  \hyperref[DC]{\color{blue}DC}  
&  \hyperref[DCHI]{\color{blue}DCHI}  
&  \hyperref[P]{\color{blue}P}  
&  \hyperref[SRC]{\color{blue}SRC}  
&  \hyperref[SRHI]{\color{blue}SRHI}  
\\ \hline \hline
                 \label{0}0  &  \hyperref[2]{\color{blue}2}  & Review Retro Planning &  
                  \hyperref[L.Q.]{\color{blue}L.Q.} \newline  \hyperref[E.Q.]{\color{blue}E.Q.} \newline  \hyperref[M.F.]{\color{blue}M.F.}  & 06/12/23 11:00  &  & -2 &  &  &  & \\ \hline 

                 \label{1}1  &  \hyperref[2]{\color{blue}2}  & Retro-review planning sprint 70 &  
                  \hyperref[E.Q.]{\color{blue}E.Q.} \newline  \hyperref[M.F.]{\color{blue}M.F.}  & 22/12/23 09:00  &  &  &  &  & 2 & \\ \hline 

                 \label{2}2  &  \hyperref[6]{\color{blue}6}  & Validar contra odoo anttes de crear una venta al recibir token &  
                 & 04/12/23 09:00  & 3 &  &  &  &  & \\ \hline 

                 \label{3}3  &  \hyperref[14]{\color{blue}14}  & Reunion Integracion Mpay para Cuadratura MGO &  
                 & 04/12/23 09:00  &  & 1.5 &  &  &  & \\ \hline 

                 \label{4}4  &  \hyperref[14]{\color{blue}14}  & Despliegue DS: release 1.3.8 &  
                  \hyperref[M.F.]{\color{blue}M.F.} \newline  \hyperref[V.V.]{\color{blue}V.V.} \newline  \hyperref[L.A.]{\color{blue}L.A.}  & 20/12/23 09:00  &  &  & 1 &  &  & \\ \hline 

                 \label{5}5  &  \hyperref[18]{\color{blue}18}  & Preparar merge de cron remotos con develop &  
                 & 27/12/23 09:00  & 5 &  &  &  &  & \\ \hline 

                 \label{6}6  &  \hyperref[19]{\color{blue}19}  & validar relacion cuponPlan al ser consultado y utilizado por usuario &  
                 & 05/12/23 09:00  & 3 &  &  &  &  & \\ \hline 

                 \label{7}7  &  \hyperref[20]{\color{blue}20}  & fix de url\_to\_pay\_subscription al utilizar transaccion antigua para crear cambio de plan &  
                 & 03/12/23 09:00  & 2 &  &  &  &  & \\ \hline 

                 \label{8}8  &  \hyperref[20]{\color{blue}20}  & Tests y caducidad para cambio de planes &  
                 & 05/12/23 09:00  & 6 &  &  &  &  & \\ \hline 

                 \label{9}9  &  \hyperref[20]{\color{blue}20}  & Implementacion conjunta api-pay pay-web para flujo cambio de planes mensual --> anual &  
                  \hyperref[M.F.]{\color{blue}M.F.}  & 13/12/23 12:00  &  & 4 &  &  &  & \\ \hline 

                 \label{10}10  &  \hyperref[20]{\color{blue}20}  & Control para que subcripcion no pueda cambiar antes de recibir un pago &  
                 & 13/12/23 09:00  & 6 &  &  &  &  & \\ \hline 

                 \label{11}11  &  \hyperref[20]{\color{blue}20}  & ajuste de endpoints abiertos &  
                 & 17/12/23 09:00  & 2 &  &  &  &  & \\ \hline 

                 \label{12}12  &  \hyperref[20]{\color{blue}20}  & Agregar atributo show\_on\_plans a planes que se muestran en endpoint publico &  
                 & 28/12/23 09:00  & 4 &  &  &  &  & \\ \hline 

                 \label{13}13  &  \hyperref[21]{\color{blue}21}  & preparar deploy 1.12.0 1.12.1 &  
                 & 09/12/23 09:00  & 7 &  &  &  &  & \\ \hline 

                 \label{14}14  &  \hyperref[21]{\color{blue}21}  & Deploy api-pay 1.12.3 &  
                 & 11/12/23 03:00  &  &  & 2.5 &  &  & \\ \hline 

                 \label{15}15  &  \hyperref[23]{\color{blue}23}  & Inicio de sistema de dominios &  
                 & 14/12/23 09:00  & 5 &  &  &  &  & \\ \hline 

                 \label{16}16  &  \hyperref[23]{\color{blue}23}  & Servicios y funcionalidad para sistema de dominios &  
                 & 16/12/23 09:00  & 4 &  &  &  &  & \\ \hline 

                 \label{17}17  &  \hyperref[23]{\color{blue}23}  & ajustes de funcionalidad para dominios &  
                 & 17/12/23 09:00  & 3 &  &  &  &  & \\ \hline 

                 \label{18}18  &  \hyperref[23]{\color{blue}23}  & Finalizar test e implementacion &  
                 & 18/12/23 09:00  & 6 &  &  &  &  & \\ \hline 

                 \label{19}19  &  \hyperref[23]{\color{blue}23}  & Script para asignar susbcripciones de dominio a usuarios ya creados &  
                 & 19/12/23 09:00  & 5 &  &  &  &  & \\ \hline 

    \end{longtable} 
 \newpage
\section{Personas}
Listado de personas y su abreviación para tareas en las que estuvieron involucrados.
\begin{table}[htbp]
    \centering
    \begin{tabular}{|p{6cm}|c|}
        \hline
        \centering{\textbf{Nombre}} & \textbf{abreviacion} \\ \hline
        Luiz Quelvez &  \label{L.Q.}L.Q.  \\ \hline
        Erik Queirolo &  \label{E.Q.}E.Q.  \\ \hline
        Marco Farias &  \label{M.F.}M.F.  \\ \hline
        Victor Vilches &  \label{V.V.}V.V.  \\ \hline
        Luis Antonio Cifuentes Loyola &  \label{L.A.}L.A.  \\ \hline
    \end{tabular}
\end{table} 
 
\section{Tarifas}
    En la siguiente tabla se presentan las abreviaciones de cada tarifa.
\begin{table}[htbp]
    \centering
    \begin{tabular}{|p{6cm}|c|}
        \hline
        \centering{\textbf{Nombre}} & \textbf{abreviacion} \\ \hline  Desarrollo &  \label{D}D  \\ \hline 
 Desarrollos en conjunto &  \label{DC}DC  \\ \hline 
 Desarrollos en conjunto en horario inhábil &  \label{DCHI}DCHI  \\ \hline 
 Presencial &  \label{P}P  \\ \hline 
 Soporte y reuniones en conjunto &  \label{SRC}SRC  \\ \hline 
 Soporte y reuniones horario inhabil &  \label{SRHI}SRHI  \\ \hline 

    \end{tabular}
\end{table} 
 \end{document}