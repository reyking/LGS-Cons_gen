\documentclass{article}
\usepackage[margin=2cm]{geometry}
\usepackage{hhline} % For highlighting table borders
\usepackage{hyperref}
\usepackage{array}
\usepackage{xcolor}
\usepackage{longtable,makecell,multirow}
\begin{document}
\section*{Resumen de actividades realizadas febrero/2024}

En el siguiente reporte se presentan las actividades realizadas por Hans Chritopher Raddatz Garcia durante el mes de febrero del 2024, donde se numeran las actividades principales dentro de la tabla de "Historias" con su total de horas, según las tarifas previamente acordadas, estas tareas principales se subdividen en cada trabajo realizado y estos son descritos en la tabla de "Tareas y detalles" indicando una descripción breve de lo realizado, las personas que estuvieron presente si es que las hubo y el tiempo en horas según cada tarifa.
Finalmente, se presenta las simbologías de tarifas y personas, además cada texto en azul implica que es un vínculo a su respectiva definición dentro de otra tabla, el cual además es clickeable. 

\section{Historias desarrolladas}
En la siguiente tabla se presentan las historias o desarrollos principales con su nombre y descripción y su tiempo total por cada tipo de tarifa.

\begin{table}[htbp]
    \centering
    \begin{tabular}{|c|p{3cm}|p{5.8cm}|m{4.5cm}|}
        \hline
        \textbf{N°} & \centering{\textbf{Nombre Historia}} & \centering{\textbf{Detalle}} & \textbf{Tiempo Total [hrs]}
        \\ \hline
  \label{2}2  & Review-retro planing & reunion bimensual para oranizar tareas del sprint &
            \begin{tabular}{m{4cm}}
Soporte y reuniones en conjunto = 1.5 \\ 

            \end{tabular} 
            \\ \hline
  \label{3}3  & Configuracion/deploy de ambientes & Todo trabajo relacionado a organizar, gestionar o realizar algun deploy en algun ambiente &
            \begin{tabular}{m{4cm}}
Desarrollos en conjunto en horario inhábil = 4.5 \\ 
\hline \hline
Desarrollos en conjunto = 0 \\ 

            \end{tabular} 
            \\ \hline
  \label{6}6  & Hotfix, revision y solucion de errores & Solucion de problemas no programados en historias &
            \begin{tabular}{m{4cm}}
Desarrollos en conjunto = 20 \\ 
\hline \hline
Desarrollos en conjunto en horario inhábil = 3 \\ 
\hline \hline
Desarrollo = 12 \\ 

            \end{tabular} 
            \\ \hline
  \label{24}24  & redis cache & cache de redis en api pay &
            \begin{tabular}{m{4cm}}
Desarrollos en conjunto en horario inhábil = 2 \\ 

            \end{tabular} 
            \\ \hline
  \label{25}25  & transaformar async await a callback & transaformar todo async-await dentro de la api-pay a callback para evitar los del event-loop &
            \begin{tabular}{m{4cm}}
Desarrollos en conjunto = 2.5 \\ 
\hline \hline
Desarrollos en conjunto en horario inhábil = 2 \\ 

            \end{tabular} 
            \\ \hline
  \label{26}26  & separar containers cron/api-pay & separar containers: cron/api-pay &
            \begin{tabular}{m{4cm}}
Desarrollos en conjunto en horario inhábil = 2.5 \\ 

            \end{tabular} 
            \\ \hline
  \label{27}27  & Partidos meagaGO & Tareas relacionada con la preparacion, Desarrollo y/o implementaciones para la visualizacion de partidos en megaGO &
            \begin{tabular}{m{4cm}}
Soporte y reuniones horario inhabil = 1.25 \\ 

            \end{tabular} 
            \\ \hline
  \label{28}28  & Revision de servicios api-pay & revisra las consultas gerneradas hacia la base de datos y su consistencia con los modelos entregados &
            \begin{tabular}{m{4cm}}
Desarrollos en conjunto = 2 \\ 
\hline \hline
Desarrollos en conjunto en horario inhábil = 1.5 \\ 

            \end{tabular} 
            \\ \hline
  \label{29}29  & Registro de correos enviados & registrar los correos enviados &
            \begin{tabular}{m{4cm}}
Desarrollo = 12 \\ 

            \end{tabular} 
            \\ \hline
    \end{tabular}
\end{table} 
 
\section{Tareas y detalles}
A continuación se presentan las tareas realizadas con su respectiva explicación, número de historia, un acrónimo de o las personas involucradas en la realización de la tarea si aplica y finalmente un detalle de horas totales trabajadas separado por cada tipo de tarifa.

\begin{longtable}{|m{0.5cm}|m{1.2cm}|p{5cm}|m{1.5cm}|m{1.5cm}||c|c|c|c|c|c|    |}
        \hline
        \multirow{2}{=}{\centering{\textbf{N°}}} & \multirow{2}{=}{\centering{\textbf{N°Hist}}} & \multirow{2}{=}{\centering{\textbf{Detalle Tarea}}}  & \multirow{2}{=}{\textbf{Personas}} & \multirow{2}{=}{\textbf{Fecha}} &   
        \multicolumn{6}{c|}{
            \textbf{Horas trabajadas [hrs]}
        } \\ 
        \hhline{~~~~~----}
        &&&&&  \hyperref[D]{\color{blue}D}  
&  \hyperref[DC]{\color{blue}DC}  
&  \hyperref[DCHI]{\color{blue}DCHI}  
&  \hyperref[P]{\color{blue}P}  
&  \hyperref[SRC]{\color{blue}SRC}  
&  \hyperref[SRHI]{\color{blue}SRHI}  
\\ \hline \hline
                 \label{0}0  &  \hyperref[2]{\color{blue}2}  & Retro review planing &  
                 & 20/02/24 01:00  &  &  &  &  & 1.5 & \\ \hline 

                 \label{1}1  &  \hyperref[3]{\color{blue}3}  & despliegue de api pay v 1.14.2 &  
                  \hyperref[M.F.]{\color{blue}M.F.} \newline  \hyperref[L.Q.]{\color{blue}L.Q.}  & 07/02/24 04:00  &  &  & 2 &  &  & \\ \hline 

                 \label{2}2  &  \hyperref[3]{\color{blue}3}  & despliegue de api pay v 1.14.3, hotfixes de despliegue de servicio de cron &  
                  \hyperref[M.F.]{\color{blue}M.F.} \newline  \hyperref[L.Q.]{\color{blue}L.Q.}  & 07/02/24 12:00  &  & 0 &  &  &  & \\ \hline 

                 \label{3}3  &  \hyperref[3]{\color{blue}3}  & Despliegue y fixes api-pay fixes en crons &  
                  \hyperref[M.F.]{\color{blue}M.F.}  & 23/02/24 04:00  &  &  & 2.5 &  &  & \\ \hline 

                 \label{4}4  &  \hyperref[6]{\color{blue}6}  & fix de cuando enviar ventas a ds segun estado del pago recibido &  
                  \hyperref[M.F.]{\color{blue}M.F.} \newline  \hyperref[L.Q.]{\color{blue}L.Q.}  & 07/02/24 08:00  &  & 3 &  &  &  & \\ \hline 

                 \label{5}5  &  \hyperref[6]{\color{blue}6}  & fix de validacion de cancelaciones y cron domain &  
                  \hyperref[M.F.]{\color{blue}M.F.}  & 05/02/24 09:00  &  & 4 &  &  &  & \\ \hline 

                 \label{6}6  &  \hyperref[6]{\color{blue}6}  & fixes menores y revision de casos para despliegues previo partido &  
                  \hyperref[M.F.]{\color{blue}M.F.}  & 03/02/24 09:00  &  &  & 2 &  &  & \\ \hline 

                 \label{7}7  &  \hyperref[6]{\color{blue}6}  & Fixes de servicios y cron para evitar incosnsistencias de consultas &  
                 & 11/02/24 09:00  & 6 &  &  &  &  & \\ \hline 

                 \label{8}8  &  \hyperref[6]{\color{blue}6}  & Fix planIdpRelacion que no permitia crear ususarios claro &  
                  \hyperref[M.F.]{\color{blue}M.F.}  & 19/02/24 09:00  &  & 5 &  &  &  & \\ \hline 

                 \label{9}9  &  \hyperref[6]{\color{blue}6}  & limitar consultas del cron finish subcription &  
                 & 20/02/24 09:00  & 4 &  &  &  &  & \\ \hline 

                 \label{10}10  &  \hyperref[6]{\color{blue}6}  & Preparacion y validar despliegue aapi-pay fixes en crons &  
                  \hyperref[M.F.]{\color{blue}M.F.}  & 22/02/24 09:00  &  & 2 &  &  &  & \\ \hline 

                 \label{11}11  &  \hyperref[6]{\color{blue}6}  & Fixes y deploy api-pay estabilidad del servicio de crons &  
                  \hyperref[M.F.]{\color{blue}M.F.}  & 26/02/24 12:00  &  & 4 &  &  &  & \\ \hline 

                 \label{12}12  &  \hyperref[6]{\color{blue}6}  & Continuacion fixes y deploy api-pay estabilidad del servicio de crons &  
                  \hyperref[M.F.]{\color{blue}M.F.}  & 26/02/24 04:00  &  &  & 1 &  &  & \\ \hline 

                 \label{13}13  &  \hyperref[6]{\color{blue}6}  & Script baja personas en limbo "disable &  
                 & 29/02/24 12:00  & 2 &  &  &  &  & \\ \hline 

                 \label{14}14  &  \hyperref[6]{\color{blue}6}  & Fixes de cancelacion y correo &  
                  \hyperref[M.F.]{\color{blue}M.F.}  & 29/02/24 02:00  &  & 2 &  &  &  & \\ \hline 

                 \label{15}15  &  \hyperref[24]{\color{blue}24}  & fixes en despliegue de redis &  
                  \hyperref[L.Q.]{\color{blue}L.Q.}  & 02/02/24 09:00  &  &  & 2 &  &  & \\ \hline 

                 \label{16}16  &  \hyperref[25]{\color{blue}25}  & renion lucho revisar &  
                  \hyperref[L.A.]{\color{blue}L.A.} \newline  \hyperref[M.F.]{\color{blue}M.F.} \newline  \hyperref[M.F.]{\color{blue}M.F.}  & 01/02/24 09:00  &  & 2.5 &  &  &  & \\ \hline 

                 \label{17}17  &  \hyperref[25]{\color{blue}25}  & renion lucho revisar &  
                  \hyperref[M.F.]{\color{blue}M.F.} \newline  \hyperref[L.A.]{\color{blue}L.A.}  & 01/02/24 09:00  &  &  & 2 &  &  & \\ \hline 

                 \label{18}18  &  \hyperref[26]{\color{blue}26}  & fixes de crons y preparacion de despliegue &  
                 & 02/02/24 09:00  &  &  & 2.5 &  &  & \\ \hline 

                 \label{19}19  &  \hyperref[27]{\color{blue}27}  & Preparacion y verificacion de estabilidad previo a partido &  
                  \hyperref[M.F.]{\color{blue}M.F.} \newline  \hyperref[L.Q.]{\color{blue}L.Q.}  & 18/02/24 04:00  &  &  &  &  &  & 1.25\\ \hline 

                 \label{20}20  &  \hyperref[28]{\color{blue}28}  & Preparacion y reviar desarrollos para desplieguie api-pay &  
                  \hyperref[M.F.]{\color{blue}M.F.}  & 12/02/24 09:00  &  & 2 &  &  &  & \\ \hline 

                 \label{21}21  &  \hyperref[28]{\color{blue}28}  & Continuacion: preparacion y reviar desarrollos para desplieguie api-pay &  
                  \hyperref[M.F.]{\color{blue}M.F.}  & 12/02/24 09:00  &  &  & 1.5 &  &  & \\ \hline 

                 \label{22}22  &  \hyperref[29]{\color{blue}29}  & modelo migraciones e implentaciones del registro &  
                 & 26/02/24 09:00  & 6 &  &  &  &  & \\ \hline 

                 \label{23}23  &  \hyperref[29]{\color{blue}29}  & test y fixes del registro de correos &  
                 & 27/02/24 09:00  & 6 &  &  &  &  & \\ \hline 

    \end{longtable} 
 \newpage
\section{Personas}
Listado de personas y su abreviación para tareas en las que estuvieron involucrados.
\begin{table}[htbp]
    \centering
    \begin{tabular}{|p{6cm}|c|}
        \hline
        \centering{\textbf{Nombre}} & \textbf{abreviacion} \\ \hline
        Marco Farias &  \label{M.F.}M.F.  \\ \hline
        Luiz Quelvez &  \label{L.Q.}L.Q.  \\ \hline
        Luis Antonio Cifuentes Loyola &  \label{L.A.}L.A.  \\ \hline
    \end{tabular}
\end{table} 
 
\section{Tarifas}
    En la siguiente tabla se presentan las abreviaciones de cada tarifa.
\begin{table}[htbp]
    \centering
    \begin{tabular}{|p{6cm}|c|}
        \hline
        \centering{\textbf{Nombre}} & \textbf{abreviacion} \\ \hline  Desarrollo &  \label{D}D  \\ \hline 
 Desarrollos en conjunto &  \label{DC}DC  \\ \hline 
 Desarrollos en conjunto en horario inhábil &  \label{DCHI}DCHI  \\ \hline 
 Presencial &  \label{P}P  \\ \hline 
 Soporte y reuniones en conjunto &  \label{SRC}SRC  \\ \hline 
 Soporte y reuniones horario inhabil &  \label{SRHI}SRHI  \\ \hline 

    \end{tabular}
\end{table} 
 \end{document}